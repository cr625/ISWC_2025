%%
%% This is file `sample-sigconf-authordraft.tex',
%% generated with the docstrip utility.
%%
%% The original source files were:
%%
%% samples.dtx  (with options: `all,proceedings,bibtex,authordraft')
%% 
%% IMPORTANT NOTICE:
%% 
%% For the copyright see the source file.
%% 
%% Any modified versions of this file must be renamed
%% with new filenames distinct from sample-sigconf-authordraft.tex.
%% 
%% For distribution of the original source see the terms
%% for copying and modification in the file samples.dtx.
%% 
%% This generated file may be distributed as long as the
%% original source files, as listed above, are part of the
%% same distribution. (The sources need not necessarily be
%% in the same archive or directory.)
%%
%%
%% Commands for TeXCount
%TC:macro \cite [option:text,text]
%TC:macro \citep [option:text,text]
%TC:macro \citet [option:text,text]
%TC:envir table 0 1
%TC:envir table* 0 1
%TC:envir tabular [ignore] word
%TC:envir displaymath 0 word
%TC:envir math 0 word
%TC:envir comment 0 0
%%
%% The first command in your LaTeX source must be the \documentclass
%% command.
%%
%% For submission and review of your manuscript please change the
%% command to \documentclass[manuscript, screen, review]{acmart}.
%%
%% When submitting camera ready or to TAPS, please change the command
%% to \documentclass[sigconf]{acmart} or whichever template is required
%% for your publication.
%%
%%
\documentclass[sigconf,anonymous]{acmart}
%%\documentclass[sigconf,authordraft]{acmart}
%%
%% \BibTeX command to typeset BibTeX logo in the docs
\AtBeginDocument{%
  \providecommand\BibTeX{{%
    Bib\TeX}}}

%% Rights management information.  This information is sent to you
%% when you complete the rights form.  These commands have SAMPLE
%% values in them; it is your responsibility as an author to replace
%% the commands and values with those provided to you when you
%% complete the rights form.
\setcopyright{acmlicensed}
\copyrightyear{2018}
\acmYear{2018}
\acmDOI{XXXXXXX.XXXXXXX}
%% These commands are for a PROCEEDINGS abstract or paper.
\acmConference[Conference acronym 'XX]{Make sure to enter the correct
  conference title from your rights confirmation email}{June 03--05,
  2018}{Woodstock, NY}
%%
%%  Uncomment \acmBooktitle if the title of the proceedings is different
%%  from ``Proceedings of ...''!
%%
%%\acmBooktitle{Woodstock '18: ACM Symposium on Neural Gaze Detection,
%%  June 03--05, 2018, Woodstock, NY}
\acmISBN{978-1-4503-XXXX-X/2018/06}


%%
%% Submission ID.
%% Use this when submitting an article to a sponsored event. You'll
%% receive a unique submission ID from the organizers
%% of the event, and this ID should be used as the parameter to this command.
%%\acmSubmissionID{123-A56-BU3}

%%
%% For managing citations, it is recommended to use bibliography
%% files in BibTeX format.
%%
%% You can then either use BibTeX with the ACM-Reference-Format style,
%% or BibLaTeX with the acmnumeric or acmauthoryear sytles, that include
%% support for advanced citation of software artefact from the
%% biblatex-software package, also separately available on CTAN.
%%
%% Look at the sample-*-biblatex.tex files for templates showcasing
%% the biblatex styles.
%%

%%
%% The majority of ACM publications use numbered citations and
%% references.  The command \citestyle{authoryear} switches to the
%% "author year" style.
%%
%% If you are preparing content for an event
%% sponsored by ACM SIGGRAPH, you must use the "author year" style of
%% citations and references.
%% Uncommenting
%% the next command will enable that style.
%%\citestyle{acmauthoryear}


%%
%% end of the preamble, start of the body of the document source.
\begin{document}

%%
%% The "title" command has an optional parameter,
%% allowing the author to define a "short title" to be used in page headers.
\title{Ontology-Driven Interaction with Large Language Models: A Semantic Framework for Structured Context Injection and External Reasoning}

%%
%% The "author" command and its associated commands are used to define
%% the authors and their affiliations.
%% Of note is the shared affiliation of the first two authors, and the
%% "authornote" and "authornotemark" commands
%% used to denote shared contribution to the research.
\author{Christopher B. Rauch}
\email{cr625@drexel.edu}
\orcid{0000-0003-2061-3413}
\affiliation{%
  \institution{Drexel University}
  \city{Philadelphia}
  \state{PA}
  \country{USA}
}

\author{Jane Greenberg}
\affiliation{%
  \institution{Drexel University}
  \city{Philadelphia}
  \country{PA}}
\email{larst@affiliation.org}

\author{Matt Kelly}
\affiliation{%
  \institution{Drexel University}
  \city{Philadelphia}
  \country{PA}
}


%%
%% By default, the full list of authors will be used in the page
%% headers. Often, this list is too long, and will overlap
%% other information printed in the page headers. This command allows
%% the author to define a more concise list
%% of authors' names for this purpose.
\renewcommand{\shortauthors}{Rauch et al.}

%%
%% The abstract is a short summary of the work to be presented in the
%% article.
\begin{abstract}
We present a novel Model Context Protocol (MCP) service that enables structured ontological knowledge integration with Large Language Models (LLMs). This framework addresses the challenge of providing LLMs with formalized domain-specific knowledge through a dedicated ontology service built on the Basic Formal Ontology (BFO). Our approach provides two key components: (1) an agent module that facilitates communication between LLMs and ontological structures through standardized interfaces, and (2) an ontology MCP server that represents semantic knowledge as RDF triples with persistent Internationalized Resource Identifiers (IRIs) and supports dynamic content negotiation. The system further enables temporal reasoning by extending RDF structures to represent ordered sequences of events, facilitating causal trace construction (without asserting the underlying causality). Through implementation in two applications \textit{ProEthica} \footnote{ProEthica, available at \url{https://github.com/cr625/ai-ethical-dm/}}
 and \textit{A-Proxy}\footnote{A-Proxy, available at \url{https://github.com/savingads/a-proxy}}
We demonstrate how this service can enhance structured knowledge integration across different domains. Empirical evaluation shows significant improvements in knowledge integration, reasoning consistency, and alignment with domain-specific requirements compared to unstructured approaches. Our framework contributes a generalizable method for bridging symbolic knowledge representation with statistical language models, with implications for semantic web applications, AI alignment, and domain-specific reasoning systems.
  
\end{abstract}

%%
%% The abstract is a short summary of the work to be presented in the
%% article.

%%
%% The code below is generated by the tool at http://dl.acm.org/ccs.cfm.
%% Please copy and paste the code instead of the example below.
%%
\begin{CCSXML}
<ccs2012>
   <concept>
       <concept_id>10002951.10003260.10003277.10003279</concept_id>
       <concept_desc>Information systems~Semantic web description languages</concept_desc>
       <concept_significance>500</concept_significance>
   </concept>
   <concept>
       <concept_id>10010147.10010178.10010179.10003352</concept_id>
       <concept_desc>Computing methodologies~Knowledge representation and reasoning</concept_desc>
       <concept_significance>500</concept_significance>
   </concept>
   <concept>
       <concept_id>10010147.10010178.10010187</concept_id>
       <concept_desc>Computing methodologies~Natural language processing</concept_desc>
       <concept_significance>500</concept_significance>
   </concept>
</ccs2012>
\end{CCSXML}

\ccsdesc[500]{Information systems~Semantic web description languages}
\ccsdesc[500]{Computing methodologies~Knowledge representation and reasoning}
\ccsdesc[500]{Computing methodologies~Natural language processing}

%%
%% Keywords. The author(s) should pick words that accurately describe
%% the work being presented. Separate the keywords with commas.
\keywords{ontology-driven LLMs, Model Context Protocol, semantic web, knowledge representation, external reasoning, ethical decision-making}

%% A "teaser" image appears between the author and affiliation
%% information and the body of the document, and typically spans the
%% page.
\begin{teaserfigure}
  \includegraphics[width=0.95\textwidth]{images/big_teaser}
  \caption{Ontology-driven structured interactions with large language models}
  \Description{Visual representation of ontology-driven communication between users and large language models, showing structured data flow through semantic frameworks}
  \label{fig:teaser}
\end{teaserfigure}

%%
%% This command processes the author and affiliation and title
%% information and builds the first part of the formatted document.
\maketitle
\clearpage

\begin{figure*}[htp]
  \centering
  \includegraphics[width=0.95\textwidth]{images/ontology_chat.png}
  \caption{Interactive ontology-driven conversation enabling structured knowledge access through the Model Context Protocol. The interface allows users to explore ontological concepts and relationships while maintaining formal semantic constraints during LLM interactions.}
  \label{fig:ontology_chat}
\end{figure*}



% Bibliography section (uncomment when you add references)
%\bibliographystyle{ACM-Reference-Format}
%\bibliography{references,software-references}

\end{document}
\endinput
%%
%% End of file `sigconf-anon.tex'.
